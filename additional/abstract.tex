\ThesisAbstract{%
The digitalisation of information now affects most fields of human activity.
From the social sciences to biology to physics, the volume, velocity, and variety of data exhibit exponential growth trends.
With such rates of expansion, efforts to understand and make sense of datasets of such scale, however driven and directed, progress only at an incremental pace.
The challenges are significant.
For instance, the ability to display an ever growing amount of data is physically and naturally bound by the dimensions of the average sized display.
A synergistic interplay between statistical analysis and visualisation approaches outlines a path for significant advances in the field of data exploration.
We can turn to statistics to provide principled guidance for prioritisation of information to display.
Using statistical results, and combining knowledge from the cognitive sciences, visual techniques can be used to highlight salient data attributes.

The purpose of this thesis is to explore the link between computer science, statistics, visualization, and the cognitive sciences, to define and develop more systematic approaches towards the design of glyphs.

Glyphs represent the variables of multivariate data records by mapping those variables to one or more visual channels (\eg, colour, shape, and texture).
They offer a unique, compact solution to the presentation of a large amount of multivariate information.
However, composing a meaningful, interpretable, and learnable glyph can pose a number of problems.
The first of these problems exist in the subjectivity involved in the process of data to visual channel mapping, and in the organisation of those visual channels to form the overall glyph.
Our first contribution outlines a computational technique to help systematise many of these otherwise subjective elements of the glyph design process.

For visual information compression, common patterns (motifs) in time series or graph data for example, may be replaced with more compact, visual representations.
Glyph-based techniques can provide such representations that can help users find common patterns more quickly, and at the same time, bring attention to anomalous areas of the data.
However, replacing any data with a glyph is not going to make tasks such as visual search easier.
A key problem is the selection of semantically meaningful motifs with the potential to compress large amounts of information.
A second contribution of this thesis is a computational process for systematic design of such glyph libraries and their subsequent glyphs.

A further problem in the glyph design process is in their evaluation.
Evaluation is typically a time-consuming, highly subjective process.
Moreover, domain experts are not always plentiful, therefore obtaining statistically significant evaluation results is often difficult.
A final contribution of this work is to investigate if there are areas of evaluation that can be performed  computationally.
}
\Abstract
