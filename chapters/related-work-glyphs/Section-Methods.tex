%\section{Techniques}
%Glyphs are often used to simultaneously represent multiple data variates in the same image~\cite{ward08glyphs}.

%\begin{itemize}
%	\item visual channel (color, shape, size, texture, and opacity) vs.\
%	\item visualizaiton dimensionality (2D, 2.5D, 3D)
%\end{itemize}




\subsubsection{Glyph Placement}

The placement of glyphs is a prominent visual stimuli and can be used
to convey information about the data.  In the context of information visualization, Ward~\cite{ward02glyphPlacement} categorizes placement 
strategies into \emph{data-} and \emph{structure-driven} placement.
The former is directly based on individual variables or spatial dimensions of the data, or on derived information such as principal components. 
Examples of data-driven strategies are placing the glyphs in a 2D~scatterplot or locating them aligned with the underlying data grid 
(in case of spatial data).
Structure-driven placement, on the other hand, is based on the ordering, hierarchical or other relationships of the data variables.
According to Ropinski et al.~\cite{Ropinski11glyphs} such strategies, however, are not directly applicable to medical data. 
Therefore, they suggest \emph{feature-driven} placement as an additional category, where glyphs are placed on local data features such as iso-surfaces~\cite{ropinski07surfaceglyphs, meyer-spradow08specGlyphs}.
We consider it useful to also consider \emph{user-driven} placement, 
where glyphs are manually placed to investigate the data at a certain location~\cite{deLeeuwVanWijk93probe, TreinishEffektiveWeather}.

In the context of data-driven placement~\cite{ward02glyphPlacement}, glyphs can be placed according to \emph{derived information} as well.
Dimensionality reduction approaches, for instance, aim at reducing the data dimensionality while maintaining the higher-dimen\-sional characteristics.
Such placement strategies can facilitate the perception of similar glyph shapes, which should be located close to each other.
Principal component analysis~\cite{WardGuo11shapeSpace}~(PCA) is such an example, which transforms multivariate data into an orthogonal coordinate system that is aligned with the greatest variance in the data.
Wong and Bergeron~\cite{WongBergeron97multiDimScaling} apply multi-dimensional scaling (MDS) for mapping higher-dimen\-sional data items into a lower-dimen\-sional space while preserving the dissimilarities between the items. Since MDS also maintains the higher-dimen\-sional structure of the data, it is well suitable for subsequent clustering.
With such methods, however, the semantic meaning of the glyph location is usually lost, in contrast to techniques that are based on the raw data~\cite{ward02glyphPlacement}.

\textbf{[DG11] Balanced glyph placement.}
Glyphs may overlap and form unwanted aggregations in image space, for instance, resulting from a regular data grid. 
Such aggregations should be avoided, since they may be erroneously identified as features~\cite{Ropinski11glyphs, ward02glyphPlacement}.
%Avoid unwanted glyph aggregations in image space}~\cite{Ropinski11glyphs}.
Laidlaw et al.~\cite{laidlaw98}, for instance, apply random jittering when placing brush strokes to represent DTI~data.
Kindlmann and Westin~\cite{kindlmannWestin06glyphPacking} use a particle system for densely packing superquadric glyphs 
(Figure~\ref{fig:sparseVsDense}b).
Meyer-Spradow et al.~\cite{meyer-spradow08specGlyphs} evenly distribute surface glyphs by combining a random placement with relaxation criteria.

In the context of glyph placement, the number of depicted data variables must be seen in relation to the available screen resolution (compare to DG2).
Large and complex glyphs such as the local probe~\cite{deLeeuwVanWijk93probe} can be used when only a few data points need to be visualised (or during individual exploration).
If many glyphs should be displayed in a dense manner, however, a more simple glyph may be desirable~\cite{pickettGrinstein88iconic, kindlmannWestin06glyphPacking, lie09glyphs}.

%A number of different strategies exist for placing glyphs~\cite{ward02glyphPlacement, Ropinski11glyphs}.
%[Explain them; difference between InfoVis and SciVis, 2D/3D]

%Guideline: glyph placement should avoid unwanted glyph aggregations in image space~\cite{Ropinski11glyphs} (e.g., using jittering)

%Feature-driven vs.\ dataset-driven placement (\eg, jittering, packing) vs.\ interactive (exploration-driven)

%The usage of certain glyph properties may also distort the interpretation of others.

%-------------------------------------------------------------------------
\subsection{Rendering}
\label{sec:rendering}
%-------------------------------------------------------------------------

In the final stage of the visualization pipeline (Figure~\ref{fig:pipeline}), glyphs are transferred from visualization space to the resulting image, where one has to cope with issues such as visual cluttering, depth perception, and occlusion~\cite{lie09glyphs}.
In the following, we discuss approaches such as halos, interactive slicing, or brushing.

\textbf{[DG12] Facilitate depth perception for 3D visualizations.}
%Important aspects when rendering many glyphs in a dense 3D~context are depth perception, occlusion, and visual cluttering~\cite{lie09glyphs}.
In cases where many glyphs overlap, \emph{halos} can help to enhance the depth perception and to distinguish individual glyphs from each other~\cite{lie09glyphs}.
Piringer et al.~\cite{piringer04scatterplots} and Interrante et al.~\cite{interranteGrosch98flow} use halos to emphasize discontinuity in depth and to draw the users attention towards objects.
For improving the depth perception for non-overlapping glyphs, a special colour map (called \emph{chroma depth}~\cite{toutin97chromaDepth}) can be used to represent depth.
Since colour is a dominant visual channel, however, it is questionable whether to use it for depicting depth instead of depicting a data variable.

%Finally, appropriate glyph placement~\cite{ropinskiPreim08glyphTaxonomy, ward02glyphPlacement}, interactive slicing, or filtering via brushing are strategies for dealing with occlusion and cluttering issues.

\textbf{[DG13] Avoid occlusion by interactive slicing or brushing:}
Occlusion is a major problem when reading glyphs. Therefore, it can be advantageous to employ interactive slicing or brushing.
Using a view dependent slice-based visualization, for example, glyphs that are located in front of a user-controlled plane are not displayed~\cite{lie09glyphs}.
Using linking and brushing in coordinated multiple views,
glyphs can be filtered out based on user-defined criteria~\cite{kehrer11interface}.

\textbf{[DG14] Avoid perspective projections when using glyph size to encode a data variable}~\cite{Ropinski11glyphs}.
In such cases, an orthographic projection is preferable, which supports the comparison of glyph size at different locations.





\subsection{Glyph Interaction}
\label{sec:interaction}

Interaction in glyph-based visualizations forms an important aspect in modern visual analytics.
Legg et al.~\cite{legg12matchPad} introduce such an example in sport notational analysis, by developing the MatchPad: an interactive visualization software that incorporates a series of intuitive user-interactions and a scale-adaptive layout to support data navigation.
One essential requirement in notational analysis in sport is the ability to review key video event footage.
Since glyphs have a limited encoding capacity, it would be impractical to map such data (e.g., a video clip, tracking data) entirely to a glyph.
Thus, the authors interactively link the playback of videos through glyphs to support rapid information retrieval.

The work of Chung et al.~\cite{chung13glyphSorting} extends this further by integrating focus+context techniques into glyph-based visual analytics to emphasise the perceptual orderability of attributes on glyphs.
They propose a system that incorporates a focus+context glyph-based interface to control and understand high-dimensional sorting of multivariate data. 
Selected components on the glyph are rendered in focus which adjusts and populates various sorting parameters within a linked \emph{Interactive, Multidimensional Glyph} (IMG) plot.
The IMG plot arranges the glyphs along two primary sorting axes. 
Various interactive tools are described to support user exploration which include: brushing tools for selecting glyphs, pan-and-zoom, and optional display preferences (e.g., connectivity lines) for conveying additional data.




